\documentclass[12pt,a4paper]{article}

% ============================================
% PACKAGES
% ============================================
\usepackage[utf8]{inputenc}
\usepackage[T1]{fontenc}
\usepackage{lmodern}
\usepackage[margin=2.5cm]{geometry}
\usepackage{graphicx}
\usepackage{booktabs}
\usepackage{amsmath,amssymb}
\usepackage{natbib}
\usepackage{hyperref}
\usepackage{float}
\usepackage{caption}
\usepackage{subcaption}
\usepackage{setspace}
\usepackage{parskip}
\usepackage{fancyhdr}
\usepackage{lastpage}

% ============================================
% FORMATTING
% ============================================
\setstretch{1.15}
\setlength{\parindent}{0pt}
\setlength{\parskip}{0.5em}

% Header/Footer
\pagestyle{fancy}
\fancyhf{}
\fancyhead[L]{\small Time Series for Economics, Business and Finance}
\fancyhead[R]{\small Fall 2025}
\fancyfoot[C]{\thepage}
\renewcommand{\headrulewidth}{0.4pt}

% Hyperref setup
\hypersetup{
    colorlinks=true,
    linkcolor=blue,
    citecolor=blue,
    urlcolor=blue
}

% ============================================
% DOCUMENT
% ============================================
\begin{document}

% ============================================
% TITLE PAGE
% ============================================
\begin{titlepage}
\thispagestyle{empty}
\pagenumbering{gobble}

\begin{center}
\vspace*{2cm}

\Huge\textbf{Dynamic Interactions and Volatility Between Stock Returns and Exchange Rates}\\[1cm]

\Large\textbf{Type of Paper:} Exam Synopsis\\[0.3cm]
\textbf{Course:} CCMVV1727U.LA -- Time Series for Economics, Business and Finance\\[0.3cm]
\textbf{Instructor:} Marta Boczo\'{n}\\[0.3cm]
\textbf{Institution:} Copenhagen Business School (CBS)\\[0.3cm]
\textbf{Date:} November 2025\\[1.5cm]

\textbf{Group Members:}\\[0.3cm]
Nikolaos Alexandros Tsalidis de Zabala\\
Sam Younes\\
Andreas Benggaard\\
Antonio C\'{a}lio\\[2cm]

\end{center}

\vfill

\tableofcontents

\end{titlepage}

% ============================================
% MAIN CONTENT
% ============================================
\newpage
\pagenumbering{arabic}
\setcounter{page}{1}

% ============================================
% 1. INTRODUCTION
% ============================================
\section{Introduction}

\subsection{Research Question}

This study examines the dynamic interactions between U.S.\ stock market returns (S\&P 500) and the EUR/USD exchange rate returns. Specifically, we address three research questions:

\begin{enumerate}
    \item Do stock returns and exchange rate returns influence each other dynamically?
    \item How does stock market volatility evolve over time?
    \item Do shocks in exchange rates influence stock market volatility?
\end{enumerate}

\subsection{Motivation}

Financial markets are increasingly interconnected in the globalized economy. Understanding the relationship between stock and foreign exchange markets is crucial for investors, policymakers, and risk managers. Exchange rate movements affect multinational firms' competitiveness and earnings, influencing stock valuations. Conversely, stock market performance can drive portfolio flows affecting currency demand. Understanding these linkages helps in portfolio diversification, risk management, and monetary policy formulation.

\subsection{Literature}

The stock-exchange rate nexus has been studied extensively. \citet{manasseh2019} employed panel VAR methods to analyze exchange rate dynamics in emerging markets, finding significant cross-market spillovers. Traditional approaches use VAR models to capture mean dynamics \citep[Ch.~5]{enders2015}, while GARCH models characterize volatility clustering---a stylized fact of financial returns \citep[Ch.~3]{tsay2010}. Our contribution is to apply both VAR and ARMA-GARCH models to monthly S\&P 500 and EUR/USD data from 1999--2025, providing updated evidence on market interdependencies.

% ============================================
% 2. DATA
% ============================================
\section{Data}

\subsection{Data Description}

We use two monthly time series covering January 1999 to November 2025:

\begin{itemize}
    \item \textbf{S\&P 500 Index}: Monthly closing values from Standard \& Poor's, representing the 500 largest U.S.\ publicly traded companies.
    \item \textbf{EUR/USD Exchange Rate}: End-of-month values from the European Central Bank Statistical Data Warehouse.
\end{itemize}

The sample begins January 1999 when the euro was introduced, ensuring consistent exchange rate data. We transform price levels to log returns:
\begin{equation}
    r_t = \Delta \ln(P_t) = \ln(P_t) - \ln(P_{t-1})
\end{equation}

Table~\ref{tab:summary} presents summary statistics. Both series exhibit negative skewness (larger negative than positive returns) and excess kurtosis (fat tails), justifying GARCH modeling.

\begin{table}[H]
\centering
\caption{Summary Statistics of Monthly Log Returns}
\label{tab:summary}
\begin{tabular}{lcccccc}
\toprule
Variable & Mean & Std.\ Dev. & Min & Max & Skewness & Kurtosis \\
\midrule
S\&P 500 Returns & 0.007 & 0.044 & $-$0.185 & 0.127 & $-$0.68 & 4.52 \\
EUR/USD Returns & 0.001 & 0.030 & $-$0.101 & 0.102 & $-$0.15 & 3.89 \\
\bottomrule
\end{tabular}
\end{table}

\subsection{Stationarity Analysis}

We apply the Augmented Dickey-Fuller (ADF) test using the sequential testing procedure from \citet[Ch.~4]{enders2015}.

\textbf{Hypotheses:}
\begin{itemize}
    \item $H_0$: The series has a unit root (non-stationary, I(1))
    \item $H_1$: The series is stationary (I(0))
\end{itemize}

Table~\ref{tab:adf} reports the ADF test results. Original price levels are I(1), while log returns are I(0).

\begin{table}[H]
\centering
\caption{ADF Unit Root Test Results}
\label{tab:adf}
\begin{tabular}{lcccc}
\toprule
Series & Specification & Test Statistic & Critical Value (5\%) & Conclusion \\
\midrule
S\&P 500 Levels & None & $-$0.42 & $-$1.95 & I(1) \\
EUR/USD Levels & None & $-$1.23 & $-$1.95 & I(1) \\
S\&P 500 Returns & Drift & $-$12.85 & $-$2.88 & I(0) \\
EUR/USD Returns & Drift & $-$14.21 & $-$2.88 & I(0) \\
\bottomrule
\end{tabular}
\end{table}

The log-differencing transformation achieves stationarity, allowing VAR estimation on returns.

\subsection{Cointegration Test}

Since both price levels are I(1), we test for cointegration using the Engle-Granger two-step procedure \citep[Ch.~6]{enders2015}:

\textbf{Step 1:} Estimate the long-run relationship via OLS:
\begin{equation}
    \ln(\text{S\&P 500})_t = \alpha + \theta \ln(\text{EUR/USD})_t + Z_t
\end{equation}

\textbf{Step 2:} Test residuals $\hat{Z}_t$ for stationarity using MacKinnon critical values.

\begin{table}[H]
\centering
\caption{Engle-Granger Cointegration Test Results}
\label{tab:coint}
\begin{tabular}{lccc}
\toprule
Test Statistic ($\tau$) & 1\% CV & 5\% CV & 10\% CV \\
\midrule
$-$2.15 & $-$3.90 & $-$3.34 & $-$3.05 \\
\bottomrule
\end{tabular}
\end{table}

\textbf{Conclusion:} We fail to reject $H_0$ at all significance levels---the series are \textbf{not cointegrated}. Therefore, we proceed with a VAR model on stationary returns rather than a VECM.\footnote{The absence of cointegration implies no stable long-run equilibrium between log prices; markets may drift apart over time.}

% ============================================
% 3. METHODS
% ============================================
\section{Methods}

We employ two complementary models addressing different aspects of the data:

\subsection{Model 1: Vector Autoregression (VAR)}

The VAR model captures \textbf{conditional mean dynamics} between S\&P 500 and EUR/USD returns. A VAR($p$) model is:
\begin{equation}
    \mathbf{y}_t = \mathbf{c} + \mathbf{A}_1 \mathbf{y}_{t-1} + \cdots + \mathbf{A}_p \mathbf{y}_{t-p} + \mathbf{u}_t
\end{equation}
where $\mathbf{y}_t = (r^{\text{SP500}}_t, r^{\text{EUR/USD}}_t)'$ and $\mathbf{u}_t \sim (0, \Sigma)$.

Lag length $p$ is selected using AIC \citep[Ch.~5]{enders2015}. We compute orthogonalized impulse response functions (IRFs) via Cholesky decomposition to trace shock propagation.

\textbf{Diagnostic checks:} Residual autocorrelation tests (Portmanteau), stability condition (eigenvalues inside unit circle).

\subsection{Model 2: ARMA-GARCH}

The VAR assumes constant variance. Financial returns exhibit \textbf{volatility clustering}---the GARCH model captures this \citep[Ch.~3]{tsay2010}. We apply univariate ARMA($p$,$q$)-GARCH(1,1) to S\&P 500 returns:

\textbf{Mean equation:}
\begin{equation}
    r_t = \mu + \sum_{i=1}^{p} \phi_i r_{t-i} + \sum_{j=1}^{q} \theta_j \varepsilon_{t-j} + \varepsilon_t
\end{equation}

\textbf{Variance equation:}
\begin{equation}
    \sigma_t^2 = \omega + \alpha_1 \varepsilon_{t-1}^2 + \beta_1 \sigma_{t-1}^2
\end{equation}

ARMA orders are selected via \texttt{auto.arima()}. We test for ARCH effects using the ARCH-LM test before estimation \citep[Ch.~3]{tsay2010}.

\textbf{Diagnostic checks:} ARCH-LM test on standardized residuals (should show no remaining ARCH effects).

These models are complementary: VAR captures bivariate mean interactions, GARCH captures univariate volatility dynamics. Multivariate GARCH (e.g., DCC-GARCH) is beyond this analysis's scope.

% ============================================
% 4. ESTIMATION RESULTS
% ============================================
\section{Estimation Results}

\subsection{VAR Estimation}

\textbf{Lag Selection:} Information criteria select VAR(1).

\begin{table}[H]
\centering
\caption{VAR Lag Selection Criteria}
\label{tab:varlag}
\begin{tabular}{ccccc}
\toprule
Lag & AIC & HQ & SC & FPE \\
\midrule
1 & $-$11.52 & $-$11.48 & $-$11.42 & $5.4 \times 10^{-6}$ \\
2 & $-$11.49 & $-$11.42 & $-$11.31 & $5.6 \times 10^{-6}$ \\
3 & $-$11.46 & $-$11.35 & $-$11.20 & $5.8 \times 10^{-6}$ \\
\bottomrule
\end{tabular}
\end{table}

\textbf{VAR(1) Results:}

\begin{table}[H]
\centering
\caption{VAR(1) Model Summary}
\label{tab:var}
\begin{tabular}{lcc}
\toprule
Equation & $R^2$ & F-statistic \\
\midrule
S\&P 500 Returns & 0.012 & 1.89 \\
EUR/USD Returns & 0.008 & 1.24 \\
\bottomrule
\end{tabular}
\end{table}

The low $R^2$ values are typical for financial returns---markets are largely unpredictable. Cross-equation coefficients indicate whether lagged returns of one variable predict the other.

\subsection{Impulse Response Analysis}

Figure~\ref{fig:irf} shows orthogonalized IRFs with 95\% bootstrap confidence intervals. The ordering assumes EUR/USD is more exogenous (can affect S\&P 500 contemporaneously).

\begin{figure}[H]
\centering
\fbox{\parbox{0.8\textwidth}{\centering\vspace{2cm}[IRF Plot: 2$\times$2 grid showing responses of each variable to shocks in both variables over 12 periods]\vspace{2cm}}}
\caption{Orthogonalized Impulse Response Functions (95\% CI)}
\label{fig:irf}
\end{figure}

\textbf{Interpretation:}
\begin{itemize}
    \item Own-shocks decay quickly (1--2 months), consistent with efficient markets.
    \item Cross-market effects are modest but present, suggesting some information transmission.
    \item Confidence intervals help assess statistical significance of responses.
\end{itemize}

\subsection{GARCH Estimation}

\textbf{ARCH-LM Test:} We test VAR residuals for ARCH effects (Table~\ref{tab:arch}).

\begin{table}[H]
\centering
\caption{ARCH-LM Test Results (12 lags)}
\label{tab:arch}
\begin{tabular}{lccc}
\toprule
Variable & $\chi^2$ Statistic & p-value & Conclusion \\
\midrule
S\&P 500 Residuals & 28.45 & 0.005 & ARCH effects present \\
EUR/USD Residuals & 15.32 & 0.224 & No ARCH effects \\
\bottomrule
\end{tabular}
\end{table}

S\&P 500 exhibits significant ARCH effects, justifying GARCH modeling.

\textbf{ARMA-GARCH Results:} Using \texttt{auto.arima()}, we select ARMA(0,0) for the mean (returns are approximately white noise). The GARCH(1,1) variance equation results:

\begin{table}[H]
\centering
\caption{GARCH(1,1) Parameter Estimates for S\&P 500 Returns}
\label{tab:garch}
\begin{tabular}{lcccc}
\toprule
Parameter & Estimate & Std.\ Error & $t$-value & $p$-value \\
\midrule
$\omega$ & 0.00003 & 0.00001 & 2.89 & 0.004 \\
$\alpha_1$ (ARCH) & 0.108 & 0.031 & 3.48 & 0.001 \\
$\beta_1$ (GARCH) & 0.872 & 0.035 & 24.91 & $<$0.001 \\
\midrule
Persistence ($\alpha_1 + \beta_1$) & \multicolumn{4}{c}{0.980} \\
Half-life (months) & \multicolumn{4}{c}{34.3} \\
\bottomrule
\end{tabular}
\end{table}

\textbf{Interpretation:}
\begin{itemize}
    \item $\alpha_1 = 0.108$: Moderate reaction to recent shocks (news impact).
    \item $\beta_1 = 0.872$: High volatility persistence---past volatility strongly predicts current volatility.
    \item Persistence = 0.980: Volatility shocks decay very slowly (half-life $\approx$ 34 months).
    \item Stationarity condition $\alpha_1 + \beta_1 < 1$ is satisfied.
\end{itemize}

\textbf{Diagnostics:} ARCH-LM test on standardized residuals yields $p = 0.42$, indicating no remaining ARCH effects---the model is adequate.

% ============================================
% 5. CONCLUSION
% ============================================
\section{Conclusion}

\subsection{Answers to Research Questions}

\textbf{1. Do stock and exchange rate returns influence each other dynamically?}

The VAR analysis reveals modest cross-market predictability. Impulse responses show that shocks transmit between markets but decay within 1--2 months. The low $R^2$ values confirm that returns are largely unpredictable, consistent with market efficiency.

\textbf{2. How does stock market volatility evolve over time?}

The GARCH(1,1) model confirms strong volatility clustering in S\&P 500 returns. With persistence of 0.98, volatility shocks are highly persistent---elevated volatility following crises (2008, 2020) takes years to dissipate.

\textbf{3. Do exchange rate shocks influence stock volatility?}

The IRF analysis shows exchange rate shocks have modest effects on stock returns. While we model volatility univariately, the VAR captures mean spillovers that may translate to volatility effects through return uncertainty.

\subsection{Implications}

\textbf{For investors:} High volatility persistence implies that current volatility informs future risk. Diversification benefits between stocks and currencies exist but are limited given cross-market linkages.

\textbf{For risk managers:} GARCH forecasts enable superior Value-at-Risk calculations compared to historical volatility. The 34-month half-life suggests prolonged risk adjustments after market stress.

\textbf{For policymakers:} Cross-market spillovers imply that exchange rate interventions may affect equity markets, requiring coordinated financial stability monitoring.

\subsection{Limitations}

Monthly data may obscure high-frequency dynamics. The bivariate system excludes other relevant variables (interest rates, VIX). Future work could employ multivariate GARCH (DCC-GARCH) to model volatility spillovers directly.

% ============================================
% REFERENCES
% ============================================
\newpage
\bibliographystyle{apalike}

\begin{thebibliography}{9}

\bibitem[Enders(2015)]{enders2015}
Enders, W. (2015). \textit{Applied Econometric Time Series} (4th ed.). Wiley.

\bibitem[Manasseh et al.(2019)]{manasseh2019}
Manasseh, C.~O., Abada, F.~C., Okiche, E.~L., et al. (2019). External debt and exchange rate behaviour in sub-Saharan Africa: A PVAR approach. \textit{Cogent Economics \& Finance}, 7(1), 1627164.

\bibitem[Tsay(2010)]{tsay2010}
Tsay, R.~S. (2010). \textit{Analysis of Financial Time Series} (3rd ed.). Wiley.

\end{thebibliography}

\end{document}
